\documentclass{beamer}

\usepackage[beamer]{shortcut}
\graphicspath{{./images/}}

\def\biblio{
    \nobibliography{../../library}
    \def\biblio{}
}

\institute{INRIA Saclay}
\author{Thomas Moreau}
\title{
    Curriculum learning
}

\setbeamertemplate{title page}[frame]
\def\extraLogo{}

\begin{document}

    \begin{frame}
        \titlepage
        \biblio{}
    \end{frame}

    \begin{frame}{Deep learning training with SGD}

        Training a deep model $f_\theta$:
       \[
            \min_\theta l(\theta) = \mathbb E \mathcal L(y, f_\theta(X))
       \]

       \vskip2em
       {\centering \Large Holistic algorithm: \textbf{Stochastic Gradient Descent}}
       \vskip2em
       \begin{enumerate}
        \item Sample a mini-batch of samples $\mathcal B$.
        \item Compute the gradient $\nabla_\theta l$
        \item Update the parameters $\theta -= \gamma\nabla_\theta l$.
       \end{enumerate}
       \vskip2em
       how do you choose the mini-batch $\mathcal B$?
    \end{frame}

    \frame{
        \frametitle{Curriculum Learning}

        Theory approach: random sampling\\[.5em]
        \small\qquad Take mini-batch at random, with replacement.\\[2em]

        Classic approach: random reshuffling\\[.5em]
        \small\qquad Take mini-batch at random, with replacement.\\[2em]

        Novel Idea: Curriculum learning\\[.5em]
        \small \qquad See easy samples first and increase the complexity as the training goes on.
    }

    \frame{
        \frametitle{Curriculum Learning}

        How to select easy/hard samples?

        \begin{itemize}
            \item Expert knowledge: can use prior knowledge from your
        \end{itemize}
    }

\end{document}
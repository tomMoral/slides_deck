\documentclass{beamer}
\usepackage{transparent}
\usepackage[beamer]{shortcut}


\usepackage[autoplay]{animate}
\usepackage{bibentry}
\usepackage{subcaption}
\usepackage{appendixnumberbeamer}

\graphicspath{{./images/}}
\def\TikzLocation{./tikz/}
\def\tkzscl{1}

\def\twocols{}
%\def\bimplies{}
%\def\partintro{}


\definecolor{primary}{RGB}{191,213,219}
\definecolor{secondary}{RGB}{144,106,66}
\setbeamercolor{block title}{fg=darkred}
\newcommand{\btitle}[1]{{\usebeamerfont{block title}\usebeamercolor[fg]{block title} #1}}

\AtBeginSection[]
{
}


\makeatletter
\def\beamer@newblock{%
  \usebeamercolor[fg]{bibliography entry author}%
  \usebeamerfont{bibliography entry author}%
  \usebeamertemplate{bibliography entry author}%
  \def\newblock{%
    \usebeamercolor[fg]{bibliography entry title}%
    \usebeamerfont{bibliography entry title}%
    \usebeamertemplate{bibliography entry title}%
    \def\newblock{%
      \usebeamercolor[fg]{bibliography entry location}%
      \usebeamerfont{bibliography entry location}%
      \usebeamertemplate{bibliography entry location}%
      \def\newblock{%
        \usebeamercolor[fg]{bibliography entry note}%
        \usebeamerfont{bibliography entry note}%
        \usebeamertemplate{bibliography entry note}}}}%
  \leavevmode\setbox\beamer@tempbox=\hbox{}\ht\beamer@tempbox=0em\box\beamer@tempbox}
  \setbeamertemplate{bibliography entry title}{}{}

\makeatother

\usepackage[square, authoryear]{natbib}


%-----------------------------------------------------------------------------
%	CUSTOM COMANDS
%-----------------------------------------------------------------------------

\def\keypoint#1{\hspace{0pt plus 1 filll}\textcolor{gray}{#1}}
\def\mycite#1{\keypoint{\small\citep{#1}}}
\def\citeconf#1#2{
    {\textcolor{gray}[}%
        {\color{linkcolor}\citealt{#1}, #2}%
    {\textcolor{gray}]}}
\def\citeconfright#1#2{\hspace{0pt plus 1 filll}{\small\citeconf{#1}{#2}}}
\def\biblio{
	\nobibliography{../../library}
	\def\biblio{}
}

\def\myitem{\hskip1em{\color{linkcolor} $\blacktriangleright$}\hskip.3em}




%\usepackage{lxfonts}

\institute{-- Concours CRCN INRIA Saclay}
\author{Thomas Moreau}
\title{Apprentissage non-supervisé pour les signaux temporels}


\setbeamertemplate{title page}[frame]


\begin{document}

\begin{frame}[plain]
\titlepage
\biblio{}
\end{frame}

\def\biblio{}

%===========================================================================
\section{Me}
%===========================================================================

\begin{frame}[t]{Parcours}

\begin{columns}[c]
    \column{1em}
    \rotatebox{90}{\Large Formation}
    \column{.9\textwidth}
\begin{list}{}{\leftmargin=0em \itemsep=0em}
    \item 2010-2014: Ecole Polytechnique - Math/Info
    \item 2013-2014: Double diplome Telecom paristech - Master MVA
\end{list}
    \begin{beamercolorbox}[rounded=true, shadow=true]{title}
        Maths appliquées, Informatique, Traitement du signal et des images.
    \end{beamercolorbox}
\end{columns}

\vskip1em

\begin{columns}[c]
    \column{1em}
    \rotatebox{90}{\Large Recherche}
    \column{.9\textwidth}
    \begin{list}{}{\leftmargin=0em \itemsep=1em}
        \item 2014-2017: Thèse -- ENS Cachan (N. Vayatis \& L. Oudre)
        \vskip-.5em
        \begin{columns}[T]
            \column{.45\textwidth}
        \begin{block}{Technique}\footnotesize
            Optimization distribuée\\
            Apprentissage non-supervisé\\
            Apprentissage de représentation
        \end{block}
        \column{.01\textwidth}
        \column{.45\textwidth}
        \begin{block}{Données}\footnotesize
            Signaux temporels multivariés\\
            Analyse de la marche\\
            Analyses oculomotrices
        \end{block}
    
        \end{columns}
        
        \item 2018-2019: Post-doctorat -- INRIA (équipe Parietal)
        \begin{beamercolorbox}[rounded=true, shadow=true]{title}
            Apprentissage non-supervisé pour les signaux en neuroscience.\\
        \end{beamercolorbox}
    \end{list}
\end{columns}

\end{frame}


\begin{frame}[t]{Contexte: Apprentissage statistique sur des signaux}
\centering
\inputTikZ{.8}{classification_tumeur}

\begin{block}{Apprentissage}
    \vskip.5em
    \begin{columns}[T]
        \column{.23\textwidth}
        \visible<1->{\myitem Supervisé}
        \column{.35\textwidth}
        \visible<3->{\myitem Faiblement Supervisé}
        \column{.35\textwidth}
        \visible<4->{\myitem Non-supervisé}
    \end{columns}
    \vskip.3em
\end{block}
\end{frame}

\begin{frame}[t]{Domaine d'application}

{
    \centering
    \vskip1em
    \only<1>{
        \begin{itemize}
            \item Exploration des signaux neurologiques de l'ECG, la MEG ou l'IRMf.\\[1em]
        \end{itemize}
        \includegraphics[height=.4\textheight]{multivariate_eeg}\\[1em]%
        
        [{\color{linkcolor} NeurIPS 2018; ICASSP 2019}]\\
        
%        \begin{itemize}
%            \item 1 article soumis en conférence,
%            \item 2 articles de journaux en cours de rédaction,
%            \item Un projet de partenariat en discussion avec B. Olshaussen (Berkeley).
%        \end{itemize}
    }
    \only<2>{
        \begin{itemize}
            \item Détéction de pas dans des signaux de marche pour la prédiction de chute chez les personnes agées.\\[1em]
        \end{itemize}
        \includegraphics[height=.4\textheight]{accelero}\\[1em]%
        
        [{\color{linkcolor} PLoS ONE 2016, Sensors 2018}]\\
        
%        \begin{itemize}
%            \item Un brevet déposé et en cours d'exploitation,
%            \item
%            \item En cours avec le laboratoire Cognac-G
%        \end{itemize}
    }
    \only<3>{
        \begin{itemize}
            \item Caractérisation de mouvements oculaires pathologiques chez les nourrissons.\\[1em]
        \end{itemize}
        \includegraphics[height=.4\textheight]{oculo}\\[1em]%
        [{\color{linkcolor} Plusieurs soumissions en cours}]\\
%        \begin{itemize}
%            \item En cours avec le laboratoire Cognac-G
%            \item Plusieurs soumissions en cours dans des revues d'oculographie.
%        \end{itemize}
    }
    \only<4>{
        \begin{itemize}
            \item Contage de cellules dans les images biologiques.\\[1em]
        \end{itemize}
        \includegraphics[height=.4\textheight]{fluospot}\\%
        
        [{\color{linkcolor}del Aguila Pla et al. 2018, IEEE TSP};
        \citealt{Yellin2017}{\color{linkcolor}, ISBI}]\\
    }
    \only<5>{
        \begin{itemize}
            \item Contage d'objet astronomiques dans des images de telescope.
        \end{itemize}
        \includegraphics[height=.4\textheight]{milky}\\%
        
        [{\color{linkcolor}del Aguila Pla et al. 2018, ICASSP};\\
        \citealt{Beckouche2013}{\color{linkcolor}, Astronomy \& Astrophysics}]\\[1em]
    }
}



\end{frame}

\begin{frame}[t]{Structure locale des signaux}
    \vskip1.5em
    \centering
    \only<1>{\includegraphics[width=\textwidth]{intro_csc_0}}%
    \only<2>{\includegraphics[width=\textwidth]{intro_csc_1}}%
    \only<3>{\includegraphics[width=\textwidth]{intro_csc_2}}%
    \only<4>{\includegraphics[width=\textwidth]{intro_csc_3}}%
    \only<5>{\includegraphics[width=\textwidth]{intro_csc_4}}%
    \only<6->{\includegraphics[width=\textwidth]{intro_csc_5}}%
    \vskip.2em
    \only<7>{%
        \includegraphics[width=.6\textwidth]{csc_explain_eq_color}
    }
    
 
\end{frame}

\begin{frame}{Challenges de l'apprentissage non-supervisé pour les signaux}


%\begin{columns}[c]
%    \column{1em}
%    %\rotatebox{90}{\Large Formation}
%    \column{.9\textwidth}
%    \begin{block}{Computationnel}
%        \myitem Parallélisation \hskip3em \myitem Structure du dictionnaire\\[.5em]
%        \keypoint{[{\color{linkcolor} ICML, 2018; preprint, 2019}]}
%    \end{block}
%\end{columns}
%
%
%\begin{columns}[c]
%\column{1em}
%\rotatebox{90}{\Large Axe 1}
%\column{.9\textwidth}
%\begin{block}{Modélisation}
%    \myitem Activations \hskip1em \myitem Motifs\\[.5em]
%    \keypoint{[{\color{linkcolor} NeurIPS, 2018; ICASSP, 2019}]}
%\end{block}
%\end{columns}
%
%
%\begin{columns}[c]
%    \column{1em}
%    \rotatebox{90}{\Large Axe 1}
%    \column{.9\textwidth}
%    \begin{block}{Théorique}
%        \myitem Évaluation des atomes \hskip1em \myitem Garanties d'estimation\\
%        \myitem Optimization globale\\[.5em]
%        \keypoint{[{\color{linkcolor} NeurIPS, 2018; ICASSP, 2019}]}
%    \end{block}
%\end{columns}


\begin{itemize}\itemsep1.5em
    \item \textbf{Computationnel:} passage à l'échelle pour les longs signaux,
    \begin{itemize}\itemsep.5em
        \item[$\bullet$] Parallélisation.%
                         \keypoint{[{\color{linkcolor} ICML, 2018; preprint, 2019}]}
        \item[$\bullet$] Utilisation de la structure du dictionnaire.%
                         \keypoint{[{\color{linkcolor} ICLR, 2017}]}
    \end{itemize}

    \item \textbf{Modélisation:} incorporer de la connaissance de domaine,
    \begin{itemize}\itemsep.5em
        \item[$\bullet$] sur les activations.%
                         \keypoint{[{\color{linkcolor} ICASSP, 2019}]}
        \item[$\bullet$] sur les motifs.%
                         \keypoint{[{\color{linkcolor} NeurIPS, 2018}]}
    \end{itemize}

    \item \textbf{Théorique:} qualité des motifs appris.
    \begin{itemize}\itemsep.5em
        \item[$\bullet$] Évaluation statistique des atomes.
        \item[$\bullet$] Garantie algorithmique de convergence globale.
        \item[$\bullet$] Garantie de reconstruction des motifs.
        \item[$\bullet$] Lien avec l'apprentissage profond.%
                         \keypoint{[{\color{linkcolor} ICLR, 2017}]}
    \end{itemize}
\end{itemize}


\end{frame}


\begin{frame}{DICOD: Optimisation distribuée pour le CDL%
              \keypoint{[{\color{linkcolor} ICML, 2018}]}}

\centering
\inputTikZ{.7}{DICOD}

\end{frame}



\begin{frame}{Application to Neuroscience \citeconfright{Dupre2018}{NeurIPS}}

\includegraphics[width=\textwidth]{rank1}

\vskip2em
\begin{itemize}\itemsep1em
    \item Good and flexible model for unsupervised learning with time-series,
    \item The patterns are "explainable", (\eg{} evoked responses, artifacts, \dots)
    \item The activations capture latency in the task-MEG responses
\end{itemize}

\end{frame}

\begin{frame}{Projet de recherche}


\begin{columns}[c]
\column{1em}
\rotatebox{90}{\Large Axe 1}
\column{.9\textwidth}
\begin{block}{Modélisation pour les signaux de neuroscience}
    \myitem Dépendance temporelle~/~Modèle multi-échelles\\[.5em]
\end{block}
\end{columns}
%
\vskip1em
\begin{columns}[c]
    \column{1em}
    \rotatebox{90}{\Large Axe 2}
    \column{.9\textwidth}
    \begin{block}{Analyse statistique des modèles convolutifs non-supervisés}
        \myitem Activations \hskip1em \myitem Motifs\\[.5em]
    \end{block}
\end{columns}
%
\vskip1em
\begin{columns}[c]
    \column{1em}
    \rotatebox{90}{\Large Axe 3}
    \column{.9\textwidth}
    \begin{block}{Apprentissage profond pour les problèmes inverses}
        
    \end{block}
\end{columns}

\end{frame}



\begin{frame}{Apprentissage non-supervisé pour les séries temporelles}
    bla
\end{frame}




%===========================================================================
\section{Conclusion}
%===========================================================================

\begin{frame}{Conclusion}
    \textbf{Convolutional Dictionary Learning}
    \begin{itemize}\itemsep.5em
        \item Flexible pattern extraction technique,
        \item Computationally tractable for more and more problems,
        \item Some application are already beginning to emerge.
    \end{itemize}
    \vskip2em
    \textbf{Challenges}
    \begin{itemize}\itemsep.5em
        \item Theoretical challenges remains (convergence, recoverability),
        \item The evaluation (and thus the parameter choices) is still not clear,
        \item Can give some insight for deep learning models?
    \end{itemize}
\end{frame}


\begin{frame}{Intégration dans l’équipe/collaborations}

\begin{columns}[c]
%    \column{1em}
%    \rotatebox{90}{}
    \column{.9\textwidth}
    \begin{beamercolorbox}[rounded=true, shadow=true]{title}
        \vskip-.1em%
        {\color{black} \bf Parietal}\\[.3em]
        \myitem{} Axe 1 : A. Gramfort, P. Ciuciu, D. Engemann\\[.3em]
        \myitem{} Axe 2 : B. Thirion, G. Varoquaux\\[.3em]
        \myitem{} Axe 3 : P. Ciuciu, A. Gramfort\\[.3em]
    \end{beamercolorbox}
\end{columns}


\vskip.5em
\begin{columns}[c]
%\column{1em}
\column{.9\textwidth}
\begin{beamercolorbox}[rounded=true, shadow=true]{title}
    \vskip-.1em%
    {\color{black}\bf National}\\[.3em]
    \myitem{} ENS Paris-Saclay (N. Vayatis) \hskip2em \myitem{} UP13 (L. Oudre)\\[.3em]
    \myitem{} Cognac-G (M. Robert, P.-P. Vidal).\\[.3em]
\end{beamercolorbox}
\end{columns}


\vskip.5em
\begin{columns}[c]
%\column{1em}
\column{.9\textwidth}
\begin{beamercolorbox}[rounded=true, shadow=true]{title}
    \vskip-.1em%
    {\color{black} \bf International:}\\[.3em]
    \myitem{} Berkeley (B. Olshausen)\\[.3em]
    \myitem{} NYU (J. Bruna).\\[.3em]
\end{beamercolorbox}
\end{columns}


\end{frame}


\begin{frame}{Profile}
\begin{columns}[T]
    \column{.33\textwidth}
    \textbf{Machine Learning}\\
    
    \begin{itemize}
        \item Unsupervised pattern recognition, Optimization
        \item 3 papers in major conferences
        \item 
    \end{itemize}
    \column{.33\textwidth}
    \textbf{Open-source Software}
    \begin{itemize}
        \item Parallel computing in python
        \item Core-developer of joblib (parallel computation engine for scikit-learn)
        \item Contributed to important OSS projects (python, scikit-learn, ...)
    \end{itemize}
    \column{.33\textwidth}
    \textbf{Applications}
    \begin{itemize}
        \item Oculographic recordings
        \item Equilibrium assessments
        \item Neuroscience
    \end{itemize}
\end{columns}
\end{frame}
%===========================================================================
% AUXILIARY SLIDES
%===========================================================================



%===========================================================================
\appendix
\section{Auxiliary Slides}
%===========================================================================


%===========================================================================
\subsection{Aux1}
%===========================================================================

\begin{frame}{Auxillary slide 1}

\end{frame}


\begin{frame}{}
\vskip2em
{\centering
    \usebeamercolor[fg]{title}
    \usebeamerfont{title}
    \Huge \bf Merci!\\[2em]}

Code available online:\\[1em]

%\includegraphics[height=.8em]{github}~\textbf{LISTA} : github.com/tommoral/AdaptiveOptim\\[1em]
\includegraphics[height=.8em]{github}~\textbf{DiCoDiLe} : \url{github.com/tommoral/dicodile}\\[1em]
\includegraphics[height=.8em]{github}~\textbf{alphacsc} :  \url{alphacsc.github.io}\\[2em]

Slides are on my web page:\\[1em]
\hskip5em\includegraphics[height=.8em]{website} \url{tommoral.github.io}
\hskip4em \includegraphics[height=.8em]{twitter} \href{https://twitter.com/tomamoral}{@tomamoral}


\end{frame}


\end{document}

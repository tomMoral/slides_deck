\documentclass{beamer}

\usepackage[beamer]{shortcut}
\graphicspath{{./images/}}

\def\biblio{
    \nobibliography{library}
    \def\biblio{}
}

\institute{INRIA Saclay}
\author{Thomas Moreau}
\title{
    EULPS: Event-based Unsupervised Learning for Physiological Signals
}

\setbeamertemplate{title page}[frame]
\setbeamercovered{invisible}
\def\extraLogo{\vskip-4em\includegraphics[height=8em]{logo_EULPS}}

\begin{document}

\begin{frame}
    \titlepage
    % \biblio{}
\end{frame}

%------------------------------------------------------------------------------

\begin{frame}{Test}
    content...
    \cite{Moreau2018}
\end{frame}

%------------------------------------------------------------------------------

\frame[t]{
    \frametitle{Principal Investigator: Thomas Moreau}

    {\bf Carrier Path:}\\[.5em]
    \highlightbox{\parbox{.9\textwidth}{
        \begin{tabular}{c p{2ex} l p{2ex} c}
            {\bf\color{black} 2014-2017}&&PhD in && ENS Cachan\\
            {\bf\color{black} 2018-2019}&&Postdoc && Inria\\
            {\bf\color{black} 2019-Present}&&Research faculty&& Inria\\

        \end{tabular}

    }}

    % tikz figure with 3 circles linked with arrows going forward and backward and the the following text in each circle:
    % - Core ML
    %- Applications
    % - Software
    \begin{tikzpicture}[overlay, remember picture]

        \node[anchor=south, xshift=-12ex, yshift=9em] at (current page.south) (cloud_ML) {\includegraphics[width=15ex]{cloud}};
        \node[anchor=north, yshift=.7em] at (cloud_ML) (ML) {Core ML\phantom{p}};
        \node[anchor=south, xshift=12ex, yshift=9em] at (current page.south) (cloud_app) {\includegraphics[width=15ex]{cloud}};
        \node[anchor=north, yshift=.3em] at (cloud_app) (App) {Applications};
        \node[anchor=south, yshift=2em] at (current page.south) (cloud_soft) {\includegraphics[width=15ex]{cloud}};
        \node[anchor=north, yshift=.7em] at (cloud_soft) (Soft) {Software\phantom{p}};

        \draw[-latex] (cloud_app) to[bend right=5] node[above,rotate=60] {} (cloud_ML);
        \draw[-latex] (cloud_ML) to[bend right=5] node[above,rotate=60] {} (cloud_app);
        \draw[-latex] (cloud_app) to[bend right=5] node[above,rotate=60] {} (cloud_soft);
        \draw[-latex] (cloud_soft) to[bend right=5] node[above,rotate=60] {} (cloud_app);
        \draw[-latex] (cloud_soft) to[bend right=5] node[above,rotate=60] {} (cloud_ML);
        \draw[-latex] (cloud_ML) to[bend right=5] node[above,rotate=60] {} (cloud_soft);

        \node[anchor=west, xshift=-1ex, yshift=3ex] at (cloud_soft.east) {\includegraphics[height=3em]{logo_benchopt}};
        \node[anchor=east, xshift=-1ex, yshift=4ex] at (cloud_soft.west) {\includegraphics[height=3em]{logo_joblib}};
        \node[anchor=east, xshift=-1ex, yshift=-3ex] at (cloud_soft.west) {\includegraphics[height=3em]{logo_loky}};
    \end{tikzpicture}

    \highlight{\parbox[c]{.18\textwidth}{
        \small 10 ML conf paper in the last years
    }}
    \hfill
    \highlight{\parbox[c]{.2\textwidth}{TEst}}\\

}

%------------------------------------------------------------------------------
\frame{
    \frametitle[t]{Recent advances in ML}

    \begin{columns}[T]
        \column{.2\textwidth}\centering
        \includegraphics[height=4em]{logo_chatGPT}\\
        \textbf{ChatGPT}
        \column{.2\textwidth}\centering
        \includegraphics[height=4em]{logo_midjourney}\\
        \textbf{Midjourney}
    \end{columns}
    \vskip2em
    {\centering \Large What do they have in common?\\}

    \vskip2em
    \pause

    \begin{columns}[T]
        \techterm{Unsupervised Pretraining}
        \visible<3>{\techterm{Global Interaction}}
    \end{columns}

    \only<2>{
        \vskip2em
        \begin{itemize}\itemindent4em
            \item Leverages large amount of unlabeled data.
            \item Learn "good" representation.
            \item Can be tweaked for various tasks.
        \end{itemize}
    }

    \only<3>{
        \vskip2em
        \begin{itemize}\itemindent4em
            \item Transformers, Self-attention, Cross attention, \dots.
            \item Capture long range interactions.
        \end{itemize}
    }


}

%------------------------------------------------------------------------------

\frame{
    \frametitle{Non-stationary signals have events}

    \begin{itemize}
        \item ECG: heart beat
        \item Gait: steps
        \item
    \end{itemize}

    \strongpoint{Building blocks of the signals}
    \strongpoint{Sparsify the signals}
}

%------------------------------------------------------------------------------

\frame{
    \bibliography{library}
}

\end{document}
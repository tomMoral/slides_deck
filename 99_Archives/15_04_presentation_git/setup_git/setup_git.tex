\documentclass[11pt]{article}

\usepackage{geometry}
\usepackage{url}
\usepackage{wasysym}

%Gummi|065|=)
\title{\textbf{Setup Git}}
\author{Thomas Moreau \& Argyris Kalogeratos \\
CMLA -- ENS Cachan}
\date{}


\begin{document}

\maketitle

\section{About}
This document provides the basic steps for installing a Git client in a Windows or 
Linux environment and connecting to the local GitLab server at CMLA.
\\
\\
\noindent\bell\ \textbf{Security note:} \emph{This document contains sensitive 
information for CMLA's computer systems. Do not send or share this file, 
or its contents, through third-party services such as email, cloud storage etc}. 

\section{Install Git client}

\subsection{Windows}
\label{sub:win}
%\begin{enumerate}
	%\item 
\hspace{1em} Install gitbash from \texttt{http://www.git­scm.com/download/win}.\\
You may use the default settings.
%\end{enumerate}
% subsection win (end) Windows
%\vspace{-.8cm}
\subsection{Linux}
\label{sub:linux}
 %\begin{enumerate}
	%\item 
	\hspace{1em} Give in the terminal \texttt{\$sudo apt-get install git}.\\
	\hspace{1em} You may use the default settings.
%\end{enumerate}
% subsection linux (end)

\section{Setup the connection to the server}
\label{sec:conn}

\noindent Launch the bash terminal and type all the lines starting with `\texttt{\$}':
\begin{enumerate}
	\item \texttt{\$ssh-keygen -t rsa}\\
	This is to create an SSH key for secure connection. 
	With the default parameters it will save you SSH in the 
	file \url{~}\texttt{/.ssh/id\_rsa.pub}.
	(Don't forget to put a passphrase when asked for, 
	it's the only way to protect it,
	if someone steal your private key, \url{~}\texttt{/.ssh/id\_rsa})
	%
	\item Copy the contents of the \url{~}\texttt{/.ssh/id\_rsa.pub}\\
	You can give \url{cat} \url{~}\texttt{/.ssh/id\_rsa.pub} 
	in the terminal to see the contents. You need to paste 
	this in the GitLab server in order to have access without 
	being requested every time for your password. 
	In Windows, the way to copy the text from the terminal window 
	is the following: right click on the title of the 
	window $\rightarrow$ Properties $\rightarrow$  
	Options $\rightarrow$ Edit Options $\rightarrow$ 
	QuickEdit Mode (put a tick). Then mark with your 
	mouse the text and press \texttt{Enter} button to copy it.
	%
	\item Go to the GitLab server at \url{https://reine.cmla.ens-cachan.fr}\\
	and then to the Profile setting $\rightarrow$ SSH Keys $\rightarrow$ 
	Add SSH Key (its a button) and paste your public key.  
	Make sure when you paste it that you don't include any 
	control characters, e.g. \url{\}\texttt{n} or \url{\}\texttt{r}.
\end{enumerate}

\noindent To be able to access the GitLab server from outside the CMLA, 
you need to have setup your Git client to contact reine using the port 2333:
\begin{enumerate}
	\item Go in your home directory, open the \texttt{.ssh} folder\\
	It might be hidden, you can reveal it with \texttt{Ctrl+H}.
 %	
	\item Create/Edit a file called \texttt{config} and add the following lines:\\
	\texttt{Host reine}\\
	    \texttt{Port 2333}\\
	    \texttt{User git}\\
	    \texttt{Hostname reine.cmla.ens-cachan.fr}
\end{enumerate}
% section conn (end)

\section{Setup the connection to the server}
\label{sec:clone}
In order to clone a repository and have a local copy of its files, 
go on the main page of that repository at the web interface of GitLab.
At the top of the page you can find the an ssh field. Use this in the 
bash \texttt{git clone put\_here\_ssh\_field}, or in any other interface.

\section{TortoiseGit}
\label{sec:tortoise}
Although for advanced use it is always better to have control on your 
repositories through terminal, it is sometimes useful to have a nice 
and easy way to do your commits and pulls. For Windows you may consider 
using TortoiseGit which is an interface to Git with context menu integration, etc. 
Find more at \texttt{http://code.google.com/p/tortoisegit}.

\end{document}
